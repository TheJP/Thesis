\begin{abstract}
    In this document different grammars to describe buildings, streets or plants were discussed, analysed and evaluated for their usefulness.

    In the Computational Planning Tools (CPlan) genetic algorithms to grow new districts and cities exist. Those genetic algorithms need fractions of street networks as chromosomes. To generate these chromosomes different clustering algorithms were implemented and tested. First the centroid-based (K-Means) algorithm was developed. Afterwards hierarchical clustering algorithms (Single-Linkage, WPGMA and UPGMA) were realised and the results were compared. To reduce the memory footprint specialised data structures were developed.

    An analysing method to select useful areas like the city centre or a business area were then created. It allows to compare different clusters and their measurements. Properties like the block area mean or the density (total area divided by total street length) are among these measurements.
\end{abstract}
