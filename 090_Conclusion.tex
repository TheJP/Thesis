\chapter{Conclusion}
In this thesis different cluster analysis algorithms were compared and implemented for graph representations of street networks.

The centroid based K-Means algorithm \ref{sec:K-Means} produced reasonable clusters. Because the used distances were vertex based (euclidean distance, not graph distance) not every cluster was a connected subgraph. To correct this error a shortest path algorithm was used \ref{sec:K-Means_shortest_path} to generate clusters which are connected subgraphs using the centroids found by the K-Means algorithm.

Hierarchical clustering \ref{sec:hierarchicalClustering} was implemented to make up for this deficit, as it can create clusters using graph distances. Hierarchical clustering was implemented with three different reduction formulae. At first hierarchical clustering using the single linkage reduction formula was implemented. It produced one huge cluster in the middle and many small ones at the border. This formula uses the minimal distances of all nodes of the compared clusters and because in a city most nodes are connected with multiple paths this result was created.

%TODO: Compare K-Means and HC?
Afterwards the reduction formulae \gls{UPGMA} and \gls{WPGMA} were implemented. Those reduction formulae proved to be a great improvement over single linkage. The created clustering were a useful division of the street network into districts. In most cases the clustering with the \gls{UPGMA} reduction formula yielded better clusters than with the use of \gls{WPGMA}.

To produce clusters which were more equal in size, the output was then modified. This means instead of splitting the hierarchy in the same order as it was created, always the biggest cluster (with the most nodes) was split. The modification leads to better balanced cluster sizes as often preferred in city clustering.

During tests with big street networks high memory usage was a problem. As a result three optimisations were made \ref{sec:memory_usage}. Firstly single-precision floating-point numbers were used. Secondly a data structure which does not store redundant values was introduced. Lastly the cached cluster distances were saved at no longer used positions of node distance matrix.

To rate the created clusters many graph analysis parameters were implemented \ref{sec:measurements-cluster-analysis}. The clusters and values can be compared and exported into a JSON-file for later use.