\pagebreak
\section{Solutions}

\subsection{Tree Creation}


\subsection{K-Means}
To separate street networks the centroid based clustering algorithm K-Mean can be used. The added implementation in CPlan \citep{cPlan:2015} is an optimised K-Means version named K-Means++. The idea of this algorithm is to select reasonable start centres to avoid many iterations.

\subsubsection{Connected Cluster Implementation} \label{sec:K-Means_shortest_path}
As described in section \ref{sec:connected_cluster_approach} the edge based distances can be calculated with a \acrlong{APSP} (\acrshort{APSP}) algorithm. Based on these distances the cluster points are then assigned. The result is a connected graph. This means every vertex can be reached from every other vertex within a cluster. In the generated figure \ref{fig:Kmeansshortestp} the artefacts described in \ref{sec:kmenasProblem} are removed.

Additional calculation time is needed for the shortest path algorithm. The differences can be compared in the section Measurements \ref{sec:measurements}.

\subsubsection{Speed Optimization}
The ETH-Zurich provided the following networks: Bad Berka (552 nodes, 626 edges), Weimar (2012 nodes, 2646 edges) and Zurich (27446 nodes, 35121 edges). The network Zurich with factor 13 more edges than Weimar resulted in long processing time. A speed improvement was  achieved by running the K-Means iterations in parallel. Every iteration can be executed free of side effects. The measurements and comparisons of the results are provided in the following section \ref{sec:measurements};
