\chapter{Introduction}
In this thesis the initial task was to learn how street networks and buildings can be described as grammar for later analysis and regeneration. There exist many different approaches like Shape Grammar \ref{sec:shape_grammar} (building faces generation), L-Systems \ref{sec:L-Systems} (plants growing), Space Syntax \ref{sec:space_syntax} and many others. 

The \textit{Chair of Information Architecture} provided a street generation and analysing tool named CPlan \ref{CPlan}. To become acquainted with the application and to allow using the existing genetic algorithms a tree generating algorithm was developed for this thesis. 

Selecting interesting areas from different cities and recombining them into a new one would allow a fast creation of new cities. To reach this aim many steps are needed. First of all the cities must be separated into reasonable parts based on vertex position or edge length. The approach of this thesis is to use clustering algorithms from the field of machine learning. Then the different area should be valued to select useful areas.

In this document centroid-based \ref{sec:K-Means} (K-Means) and hierarchical \ref{sec:hierarchicalClustering} (WPGMA, UPGMA) clustering algorithms were compared \ref{sec:measurements-speed} and extended to correct wrong assignments \ref{sec:K-Means_shortest_path} and to reduce the memory footprint \ref{sec:memory_usage}. %TODO In more details!

The implemented clusters where then measured \ref{sec:measurements} based on the suggestions \ref{sec:clusterRating} provided by the ETH-Zurich and compared \ref{sec:measurements-cluster-analysis}. Additionally the results can now be exported into a JSON-File for further use.

To recombine the separated areas/districts to a new city another project at ETH-Zurich is currently working on a regrow algorithm \ref{sec:future_work}. 