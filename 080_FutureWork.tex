\chapter{Future Work}
\label{sec:future_work}

Despite many cluster analysis approaches were tested in this thesis, other methods like distribution-based or density-based clustering exist. They could be tested and compared against the implemented versions for this thesis.

Additional features could be used to analyse the given street networks. For example the hierarchy create for the hierarchical clustering method could be used in K-Means or the edge centres could be used instead of the vertices.

To generate faster results some optimisations are possible. For the K-Means algorithm a K-D tree could be used. The current Single-Linkage algorithm $O(n^2)$ could be optimised with an data structure called quadtree with $O(n)$.

More detailed analysis of the measured clusters could be created based on urban planing data. Additional more detailed measurements methods could be added and the create results compared.

At the ETH there are other running projects to extend CPlan with more functionality. One project is to allow removing a subgraph within a city and regrow a subgraph into the gap from the edges. Another project is to recombine the created clusters by the measured results provided by this thesis.

The provided application CPlan was recently ported to 64 bit architecture. Unfortunately some libraries are still only 32 bit versions, which leads therefore to problems. For this thesis unit tests were added to the CPlan application to ensure the proper working of the new classes and functions. Additionally many tests should be added to test the core functionality of the given application. A rework of the application would allow a more efficient and faster development.