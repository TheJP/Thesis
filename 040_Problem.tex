\chapter{Problem}
The idea of \gls{FCL} is to create, analyse and understand sustainable cities. All tools created for this purpose are provided as open source. The \acrlong{iA} has made many contributions to the \gls{FCL} in recent years. In the course of which CPlan \ref{CPlan} was often used and further developed.

%With CPlan cities can be imported and block areas can be created. Many growing algorithms with random parameter exist and are fully working. To apply modification on the created cities a tree should be created from the existing street network. This would allow genetic modifications on existing networks.

%Random parameters to generate new cities are a good approach but it is not possible to use existing map data. To work with predefined subnetworks of a city map these parts have to be separated based on the given data.



The extracted parts should then be measured for their usefulness to reuse and recombine them at a new position in a new generated city.
